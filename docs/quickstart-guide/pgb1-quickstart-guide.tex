\documentclass[10pt]{article}
\usepackage{amssymb,amsmath,amsthm,amsfonts}
\usepackage{multicol,multirow}
\usepackage{calc}
\usepackage{ifthen}
\usepackage[portrait]{geometry}
\usepackage[hidelinks,colorlinks=false]{hyperref}
\usepackage{tcolorbox}
\usepackage{graphicx}
\usepackage{tikz}
\usetikzlibrary{arrows.meta,
    positioning,
    quotes, shapes, spy}
\usepackage[percent]{overpic}
\usepackage{../wnm-latex-utils}

\ifthenelse{\lengthtest { \paperwidth = 11in}}
    { \geometry{top=.5in,left=.5in,right=.5in,bottom=.5in} }
	{\ifthenelse{ \lengthtest{ \paperwidth = 297mm}}
		{\geometry{top=1cm,left=1cm,right=1cm,bottom=1cm} }
		{\geometry{top=1cm,left=1cm,right=1cm,bottom=1cm} }
	}
\pagestyle{empty}
\makeatletter
\makeatother
\setcounter{secnumdepth}{0}
\setlength{\parindent}{0pt}
\setlength{\parskip}{0pt plus 0.5ex}

\def\fulldocurl{https://weenoisemakers.com/pgb-1}
\def\firmwareupdatesurl{https://weenoisemakers.com/pgb-1}
\def\doctitle{WNM PGB-1 Quick Start Guide}

\newcommand{\OLEDscreenshot}[1]{
\begin{tcolorbox}[hbox, colframe=black, colback=white, boxsep=-3pt]%
\includegraphics{#1}
\end{tcolorbox}
}

\newcommand{\reflabel}[2]{
[\hyperref[#1]{#2}]
}

\title{\doctitle}

\begin{document}

\raggedright
\footnotesize

\begin{center}
     \Large{\textbf{\doctitle}} \\
\end{center}
\begin{multicols*}{2}
\setlength{\premulticols}{1pt}
\setlength{\postmulticols}{1pt}
\setlength{\multicolsep}{1pt}
\setlength{\columnsep}{2pt}

\section{Complete User Guide}

This document provides the essential information to start using the PGB-1.
The complete user guide is available at: \url{\fulldocurl} \qrcode[height=40pt]{\fulldocurl}

\section{Hardware Layout}

\includesvg[width=250pt]{../assets/PGB1-front-black-n-white-with-labels.svg}

\section{Powering Up}

To power up the device, slide the Power Switch to the left. The device will power up and the status LED will glow cyan blue. 

\section{Charging}

\section{Navigation}

Use \kbd{Left}, \kbd{Right}, \kbd{Up}, \kbd{Down}, \kbd{A}, and \kbd{B} buttons to navigate inside the PGB-1 menus.

The \kbd{Left} and \kbd{Right} buttons select between the different settings.
The \kbd{Up} and \kbd{Down} buttons change the value of the selected setting.
The \kbd{A} button is used for positive actions (select, accept, enable, etc.), 
the \kbd{B} button for negative actions (go back, reject, disable, etc.).

On the screen, a menu position indicator is showing how many pages of settings are available and which one is currently displayed.
In addition, a left/right facing arrow shows there are pages to the left/right of the current page. 

For example, looking at the track settings:

% LaTeX is amazing... (https://tikz.dev/library-spy)
\begin{center}
\begin{tikzpicture}
    [spy using outlines={circle, magnification=3, connect spies}, 
    image/.style = {scale=0.8,},
    ]
  
    \node[image] (A) {\OLEDscreenshot{../assets/OLED-screenshots/PGB1-OLED-track-LFO.png}};
    
    % Page indicator
    \spy [black, size=2cm] on (-0.62,0.35) in node [left] at (-2.5,-0.5);

    % Right arrow
    \spy [black, size=2cm] on (1.68, 0.45) in node [right] at (2.5,-0.5);
\end{tikzpicture}
\end{center}

\section{Firmware Updates}

\url{\firmwareupdatesurl} \qrcode[height=40pt]{\firmwareupdatesurl}

\section{Audio In/Out and Volume}

Audio output will automatically switch from internal speaker to headset output when a jack is inserted. To change the output volume, hold the \kbd{Play} button down and press or hold \kbd{Up} or \kbd{down}. Max volume can be very loud, especially with headphones or earbuds. Be careful, high volume for a long time may damage your hearing.

Audio input is disabled by default. To configure audio input, open the main menu with \kbd{Menu}, then select \pgbmenu{Inputs}. You will be able to unmute/mute for each input (Line-in, internal microphone, headset microphone), and set a common volume. On the second page. You can select a common effect for audio input.

\section{Shortcuts}

\begin{itemize}
    \item[] Increase Volume \dotfill Hold \kbd{Play} + Press \kbd{Up}
    \item[] Decrease Volume \dotfill Hold \kbd{Play} + Press \kbd{Down}
    \item[] Increase BPM \dotfill Hold \kbd{Play} + Press \kbd{Right}
    \item[] Decrease BPM \dotfill Hold \kbd{Play} + Press \kbd{Left}
    \item[] Mute Track \dotfill Hold \kbd{Play} + Press \kbd{1} to \kbd{16}
    \item[] Solo Track \dotfill Hold \kbd{Play} + Press \kbd{Track}, then Press \kbd{1} to \kbd{16}
    \item[] Enter Sampling Mode \dotfill Hold \kbd{CPY/FX} + Press \kbd{Edit}
\end{itemize}

\section{Tips and Tricks}

\section{Troubleshooting}

\subsection{Situation}

\subsubsection{Device won't turn off}

In case the device doesn't turn off when the power switch is in the \kbd{Off} position, it means there is an unrecoverable software malfunction. 

The only way out is to follow the \reflabel{procedures:reset}{Reset Procedure} below.

\subsection{Procedures}

\subsubsection{Reset Procedure}
\label{procedures:reset}

\begin{important}
Changes to the current project, if any, will not be saved.
\end{important}

Use a paperclip to press the reset button (leftmost hole at the bottom right of the front panel).

\subsubsection{Forced Firmware Update Procedure}

First, make sure the device is off (power switch to the right). If the device doesn't turn off, follow the \reflabel{procedures:reset}{Reset Procedure}.

Connect the device to your computer using a good quality USB cable.

Use a paperclip to hold the `boot` button down (rightmost hole at the bottom right of the front panel).

While holding the `boot` button down, turn on the device (power switch to the left).

\end{multicols*}

\end{document}